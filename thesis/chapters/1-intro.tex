\chapter{Introduction}
In recent years \glspl{uav} have transitioned from niche research platforms to versatile tools in a wide range of civil and industrial domains. Their ability to navigate complex environments, carry diverse payloads, and coordinate as autonomous teams has unlocked new possibilities. The applications range from rapid disaster response to precision agriculture. However, as applications grow more demanding and dynamic, conventional control methods reveal significant limitations in robustness and scalability. In particular, scenarios involving multiple \glspl{uav} collaboratively transporting a shared, cable-suspended load pose challenges that are difficult to address with traditional model-based control approaches.
This thesis presents a deep \gls{rl} framework for decentralized multi-\gls{uav} cable-suspended payload transport, aiming to learn policies that stabilize the payload, track the target and adapt to disturbances in real time.
\section{Motivation}
\glspl{uav} have seen rapid adoption across a variety of civil and industrial applications, including construction site surveying, debris cleanup after natural disasters, search and rescue operations in hazardous environments, agriculture, package delivery, and inspection or maintenance tasks at nuclear facilities \autocite{Idrissi2022AROA,Lyu2023UnmannedAVA,Chen2021FromUSA,Abbaraju2018SensingASA}. In these contexts, reliable and adaptive control of single and multiple \glspl{uav} is critical to ensure mission success and safety.

Traditional model-based control methods for \glspl{uav} require highly accurate mathematical models of the vehicle dynamics and the task environment. Such methods are often overly sensitive to modeling errors and external disturbances (e.g., wind gusts, changing payload mass, collisions), which limits their robustness in real-world scenarios. Moreover, scaling these controllers to coordinate swarms of \glspl{uav} for cooperative tasks introduces significant complexity. Centralized approaches suffer from communication bottlenecks and single points of failure, while decentralized schemes often struggle to guarantee global performance.

A particularly challenging application for \glspl{uav} is aerial payload transport. Cable-suspended payload transport offers several advantages over other attachment methods. By suspending a load beneath one or more \glspl{uav} using flexible cables, the system can adapt to varying payload shapes and sizes without the need for specialized fixtures. Cables are lightweight and easily stowed, which decreases the overall system mass and improves flight endurance. Additionally, cable-suspended configurations provide natural compliance that can absorb shocks and reduce the transmission of vibrations from the \gls{uav} rotors to the payload. These benefits make cable-based transport particularly attractive for missions requiring the delivery of sensitive equipment. However, the suspended load can exhibit significant pendulum-like oscillations, which complicate stabilization and trajectory tracking. When multiple \glspl{uav} cooperate to carry a shared suspended payload, coordination becomes even more complex, as each vehicle's motion affects the tension distribution and dynamic behavior of the cables. Traditional control approaches often rely on simplified models, which break down in aggressive maneuvers or under external disturbances (e.g., gusts of wind, abrupt payload shifts, collisions, landing and takeoff). Moreover, designing decentralized controllers that guarantee safe load distribution and robust performance across all \glspl{uav} remains an open research problem \cite{estevez_review_2024}.

\gls{rl} offers a promising alternative by leveraging data-driven policy optimization to automatically adapt to uncertainty and complex dynamics. \gls{rl} agents can learn flexible control strategies directly from interaction with a simulator or the real world, showing increased resilience to disturbances and unmodeled effects. Recent studies have demonstrated the potential of both single-\gls{uav} and multi-\gls{uav} \gls{rl} in tasks such as target tracking, formation flight, achieving performance on par or exceeding classical controllers under challenging conditions \autocite{Hwangbo2017ControlOAA, kaufmann_champion-level_2023, Song2023ReachingTL, huang_collision_2024,eschmann_learning_2024-1}.

The goal of this thesis is to develop a deep \gls{rl} approach for coordinated multi-\gls{uav} payload transport, with a focus on cable-suspended loads. By designing an efficient training pipeline that supports decentralized learning, we aim to achieve control policies that can stabilize the suspended payload, distribute load forces evenly among cooperating \glspl{uav}, and adapt to disturbances in real time, while tracking the payload's position. Such policies promise to be more robust, safer, and more autonomous than existing methods, enabling scalable multi-\gls{uav} teams to perform complex transport missions in unstructured and dynamic environments.
\section{Research Objectives}
To address the challenges outlined above, this work is guided by the following research objectives:
\begin{itemize}
    \item \textbf{Formulate the decentralized multi-\gls{uav} payload transport problem:} Define a suitable multi-agent \gls{mdp} for cable-suspended payload transport, specifying local observation and action spaces for each \gls{uav}, and determine how to coordinate through decentralized policies.
    \item \textbf{Design and implement an efficient training pipeline:} Develop a high-performance simulation framework for multiple quadrotors carrying a payload to enable scalable policy learning.
    \item \textbf{Investigate reward design:} Analyze how to decompose the overall objective into separate tracking and stability components, and study the effect of different components in the reward function.
    \item \textbf{Evaluate robustness under disturbances:} Test the learned policies in simulation scenarios, especially under harsh conditions with disturbances, to assess resilience compared to classical controllers.
    \item \textbf{Demonstrate scalability to multiple agents:} Show that the proposed decentralized approach can handle varying numbers of \glspl{uav}, maintaining performance as the team size increases.
\end{itemize}
\section{Contribution}
This thesis makes the following key contributions:
\begin{itemize}
    \item \textbf{Decentralized \gls{rl} for Cable-Suspended Payloads.} We formulate the multi-\gls{uav} payload transport problem as a decentralized multi-agent \gls{rl} task, enabling each \gls{uav} to act based on local observations while still achieving coordinated global behavior under harsh dynamic conditions.
    \item \textbf{High-Performance GPU-Parallel Training Pipeline.} We develop a GPU-parallelized framework in JAX and MJX that simulates cable-suspended payload transport, capturing the complex dynamic interactions among multiple \glspl{uav}. This pipeline is optimized for contact-rich scenarios, and future work could leverage the contact dynamics to learn aerial manipulation and other complex tasks.
    \item \textbf{Modular Reward Design.} We investigate the trade-off between tracking accuracy and system stability by decomposing the reward into distinct tracking and stability components, and we analyze how different reward weightings impact overall performance and safety.
\end{itemize}

\section{Thesis Overview}
Chapter 2 reviews related work on traditional model-based control for multi-\gls{uav} payload transport and existing reinforcement learning approaches for \gls{uav} control, including sim-to-real methods, single-\gls{uav} payload tasks, swarm \gls{marl}, and cooperative transport via RL. Chapter 3 provides the necessary background on reinforcement learning fundamentals, multi-agent \gls{mdp}, \gls{ppo} in both independent and centralized settings, and quadrotor dynamics and control. \todo{write overview}